\documentclass[12pt]{article}

%% preamble: Keep it clean; only include those you need
\usepackage{amsmath}
\usepackage[margin = 1in]{geometry}
\usepackage{graphicx}
\usepackage{booktabs}
\usepackage{natbib}

% highlighting hyper links
\usepackage[colorlinks=true, citecolor=blue]{hyperref}


%% meta data

\title{Proposal: How does Height and Weight affect NBA Player Performance?}
\author{Justin Chan\\
  \date{October 9th, 2023}\\
  University of Connecticut
}

\begin{document}
\maketitle

\section{Introduction}
In the increasingly competitive world of professional basketball, players are doing everything they can to 
gain a competitive advantage over another. With practice, coaches, and trainers players can work hard at 
mastering their craft but what about the more physical aspects of their game? How does a players' height
and weight impact the performance ability of an NBA Player? Does being taller translate to scoring more 
points per game? Does weight increase the amount of rebounds you will grab? The answer to these 
questions offers invaluable insight to coaches, scouts, and analysts in making data-driven decisions for their 
teams. As far as similar research, a study published in the British Journal of Sports Medicine looked at height
as a predictor within all sports and found that height and sporting success was highly dependent on the type
of sport \citep{tucker2012makes}. Another study done by head researcher Shaoliang Zhang looked at height and 
weight as a single predictor along with league experience \citep{zhang2017performance}. This paper would 
build onto that by looking at how player performance is affected by height and weight individually within the
sport of basketball.

\section{Specific Aims}
The research question that this paper will work towards answering is, "How does Height and Weight affect
NBA Player Performance?" In a more statistical manner, we will be answering the question of how well the 
predictors of Height and Weight can model an NBA Players' ability to score, rebound, assist as well as a variety
of other metrics. This question can also be used to help coaches, scouts, and team managers make more 
informed decisions for the well-being of their team.

\section{Data}

The dataset that I will be using is taken from Kaggle at \href{https://www.kaggle.com/datasets/justinas/nba-players-data}{NBA Players} 
which sourced its data from the official NBA website. The dataset contains 12.3 thousand observations with 
22 variables taken from 1996 to 2021. It gives the per game averages of each player per season along with
other variables including height and weight. Some of the variables include age, games  played, points per game, 
rebounds per game, assists per game, and net rating. The data has been cleaned with no rows of missing data 
or other data quality issues.

\section{Research Design and Methods}

In terms of design and methods, I will first preprocess and scan the data through creating a number of
visualizations and other methods to ensure data quality, identify outliers within the data, and to see if any of the
predictors need to be transformed in anyway. Next, I will fit a number of regression models with height and 
weight as the predictors to see how the predictors correlate with points, rebounds and assists. Finally, I'll 
use the information gathered from this to try to create the best model to predict each basketball metric using 
either one or both of the predictors.

\section{Discussion}

Some of the challenges that I would expect in this task is finding the right transformations and tuning
for each of the models in order for each them to perform to their best. Also, another difficulty could be trying to
fit other types of models to the dataset but it is unclear if that would be appropriate. If something unexpected
should occur, I will first look to online resources for help and try fix the issue myself otherwise, I will ask professors
more knowledgeable in field for their help and guidance. If it is a problem with the dataset, there are other datasets
available but I don't expect such issues. Some limitations of the work is the dataset. The current dataset only 
contains height and weight but wingspan is also another physical attribute that could affect the a players ability to 
grab a rebound for example. Another limitation, is the types of models being fitted for this paper. There could be 
more sophisticated models more fit for the situation than the relatively simple ones being used.

\bibliography{proposalbib}
\bibliographystyle{plainnat}

\end{document}